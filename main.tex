\documentclass[journal=jacsat,manuscript=article]{achemso}
\usepackage[latin1]{inputenc}
\usepackage{amsmath}
\usepackage{amsfonts}
\usepackage{amssymb}
\usepackage{graphicx}
\usepackage{numprint}
\usepackage{mathtools}
\usepackage{enumitem}
\usepackage{hyperref}
\usepackage[squaren,grey]{SIunits}
\usepackage{setspace}
\doublespacing

\author{Charles Maher, Haina Wang}
\affiliation[Princeton University]{Department of Chemistry Torquato Lab, Princeton University, Princeton, NJ 08544, USA}
\email{hainaw@princeton.edu}

\title{Summary 3-27-19}
  
 
\begin{document}
	\maketitle
	\newpage
	\section{Comparing first moment of $h(r)$ and $S(0)$}
	The integral $1+\rho\int h(r)d\mathbf{r}$ is truncated at a distance $r_t$ by inspection of the convergence of the numerical integral $\int_{0}^{x} rh(r)dr$ (usually $5\sigma\leq r_t\leq8\sigma$) and compared to $S(0)$ obtained in simulations. All simulation runs use 400 particles.
	
	\begin{center}
	\begin{tabular}{|c|c|c|c|c|c|c|}
		\hline 
		$\rho$ & $T$ & $T/T_c$ & State &S(0) by 4th virial & $1+\rho\int h(r)d\mathbf{r}$ & $S(0)$ \\ 
		\hline 
		0.0163 & 0.40 & 0.71 & Vapor & 1.35 & 1.28 & 1.28 \\ 
		\hline 
		0.02 & 0.45 & 0.8 & Vapor & 1.48 & 1.45 & 1.45 \\ 
		\hline 
		0.05 & 0.50 & 0.89 & Vapor & 2.28 & 2.05 & 2.07 \\
		\hline
		0.10 & 0.60 & 1.07 & Supercrit. & 2.12 & 1.97 & 1.95 \\
		\hline
		0.60 & 0.55 & 0.98 & Liquid & & 0.34 & 0.48 \\
		\hline
		0.60 & 0.45 & 0.8 & Liquid & & 0.5--1.0 (Oscillates) & 0.96 \\
		\hline
	\end{tabular} 
	\end{center}
	
	
	Therefore $1+\rho\int h(r)d\mathbf{r}$ agrees best with simulated $S(0)$ at low densities. When the density is higher, the integral converges only at large $r$ and is more prone to noises in the simulated $h(r)$ at large $r$.

\section{Fitting of $S(k)$ at small $k$ ($0\leq k\leq 5$)}
	The states chosen here for fitting of $S(k)$ are such that $S(k)$ monotonically decreases as $k$ decreases to 0: They are liquid-like supercritical fluids.
	
\begin{center}
		\begin{tabular}{|c|c|c|c|}
			\hline 
			$\rho$ & $T$ & $T/T_c$ & Order 3 poly. fit \\ 
			\hline 
			0.6 & 2 & 3.57 & $0.026k^3 - 0.1138k^2 + 0.1546k + 0.1219$ \\ 
			\hline 
			0.7 & 1.5 & 2.68 & $0.0363k^3 - 0.1973k^2 + 0.3155k - 0.0257$ \\ 
			\hline 
			0.7 & 0.75 & 1.34 & $0.0408k^3 - 0.2286k^2 + 0.3563k - 0.0278$ \\
			\hline
		\end{tabular} 
	\end{center}

	We note, for the moment, that first and second order polynomials give visibly very inaccurate fit of $S(k)$. The third order polynomials give $R^2>0.99$.
	
\section{Computing $S(k)$ and $g_2(r)$ using Ge's Code}
$S(k)$ seems to be more well-behaved closer to the origin, but for higher densities I still seem to be getting jagged peaks. The same follows for the low density $g_2(r)$, I don't have much intuition yet for how the binning works with that program, so I'm unsure how to address the low-$r$ behavior not being what we expect at this point in time. The large-$r$ ($r>17$) behavior for $\rho=0.765$ is also somewhat troubling, is it reasonable to truncate $g_2(r)$ much earlier than this?

\begin{figure}[H]
    \centering
    \includegraphics[width=6.0in]{PrelimGeg2.eps}
    \caption{}
    \label{fig:my_label0}
\end{figure}

\begin{figure}[H]
    \centering
    \includegraphics[width=6.0in]{PrelimGeSk.eps}
    \caption{}
    \label{fig:my_label1}
\end{figure}

\section{Literature for $\tilde{c}(k)$ or $\tilde{h}(k)$}
    
    Verlet's paper\cite{Verlet1968} talks about two types of decay of $h(r)$: exponential decay at small $\rho$ and damped oscillatory decay at  large $\rho$. The asymptotics of $h$ can be analyzed by the poles of $\tilde{h}(k)$ at small $k$.

	Dijkstra and Evans\cite{Dijkstra2000} report a curious way to ``force'' long-range information of $h(r)$ from small simulation size. They propose a way to determine if $h(r)$ has exponential decay or damped oscillatory decay, \textit{i.e.} with the transformations:
	
	 $h(r)\text{ from simulation}\rightarrow \tilde{h}(k)\rightarrow \tilde{c}(k) \rightarrow c(r)\rightarrow\text{poles of }\tilde{h}(k)\rightarrow\text{the decay type of }h(r)$.
	
	Because the integral in their second last step is truncated at $r=2.5\sigma$, the work might be only valid for tasks involving relatively small $k$, \textit{e.g.} $1\leq |k| \leq 2$ in their paper. It may not be particularly helpful to determine the behavior as $k\rightarrow 0$, for the integral $1+\rho\int h(r)d\mathbf{r}=S(0)$ usually does not converge until $r\geq 5\sigma$. However, it may be instructive to calculate some poles of $\tilde{h}(k)$?
	
	Ref. \citenum{Zhao2011} has a comparison of $1+\rho \tilde{h}(k)$ with $S(k)$ for liquid water. They also mention significant difference as $k$ go to 0, and that Fast Fourier Transform underestimates $S(0)$ as observed in the previous section.
	

\section{Various LJ Correlation Functions in the Literature}
I'm still unable to find a direct study of the structure factor, but I have found the following on other correlation functions related to the OZ formalism:

\subsection{Bridge function and cavity correlation function for the Lennard-Jones fluid from simulation}
https://doi.org/10.1063/1.463142
\begin{itemize}
    \item $y(r)$ and $h(r)$ are computed from simulation
    \item $c(r)$ determined using the OZ formalism
    \item "Our work appears to be the first complete study of the distribution functions for the LJ fluid."
    \item Only actually present $y(r)$ and $B(r)$ (the bridge function), for liquid-like densities in temperatures ranging from the triple point to supercritical temperatures
    \item "Apparently the first time that the bridge function has been calculated for any fluid other than the hard sphere"
\end{itemize}

\subsection{The potential distribution-based closures to the integral equations for liquid structure: The Lennard-Jones fliud}
https://doi.org/10.1063/1.474974
\begin{itemize}
    \item Formally, $B(r)$ is an infinite cluster expansion, the series of which is unknown
    \item That $B(r)$ is a functional is guaranteed by the laws of physics, but most closure relations make $B(r)$ a function, which is by no means is guaranteed to be a success because there is no theorem stating the uniqueness/existence of this relationship
    \item $B(r)$, $\textrm{ln}(y(r))$, $c(r)$, and $g(r)$ reported for liquid-phase conditions 
\end{itemize}

\subsection{Some more of the same:}
\begin{itemize}
\item Integral equation theory for uncharged liquids: The Lennard-Jones fluid and the bridge function (https://doi.org/10.1063/1.470724)
\end{itemize}
This paper reports more attempts at solving $B(r)$, as well as some plots of $g(r)$. It also has a phase diagram produced from their results.

\bibliographystyle{achemso}
\bibliography{StructFacLJ}	

\end{document}
